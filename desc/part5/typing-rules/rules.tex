\documentclass[a4paper]{article}


\usepackage[margin=2cm]{geometry}

\usepackage{mathpartir}
\usepackage{amssymb}
\usepackage{xcolor}
\title{\vspace{-2ex}Typing rules for MiniC\vspace{-2ex}}
\date{}

\begin{document}
\maketitle
\thispagestyle{empty}  % removes page number
 

 
  
\section{Declarations}
 
 
\begin{mathpar}
\inferrule*[left=VarDecl(\textup{v})]{
\textbf{T} \notin \{\textbf{void}\}
}{add\ \langle v : \textbf{T}\rangle\ to\ \Gamma}
\end{mathpar}



\begin{mathpar}
\inferrule*[left=FunDecl(\textup{f})]{\ }{add\ \langle f: (U_1, \ldots, U_n) \rightarrow \textbf{T} \rangle\ to\ \Gamma}
\end{mathpar}



\section{Expressions}

% Literals
\begin{mathpar}
\inferrule*[left=IntLiteral(\textup{i})]{\ }{\Gamma \vdash i : \textbf{int}}
\and
\inferrule*[left=StrLiteral(\textup{s})]{\ }{\Gamma \vdash s : \textbf{char[s.length+1]}}
\and
\inferrule*[left=ChrLiteral(\textup{c})]{\ }{\Gamma \vdash c : \textbf{char}}
\end{mathpar}

% VarExpr
\begin{mathpar}
  \inferrule*[left=VarExpr(\textup{v})]{\vdash \langle v : \textbf{T}\rangle \in \Gamma}{\Gamma \vdash v : \textbf{T}}
\end{mathpar}

% FunCallExpr
%% \begin{mathpar}
%% \inferrule*[left=FunCallExpr(\textup{f})]{
%% %
%% \vdash \langle f: \overline{U} \rightarrow \textbf{T} \rangle \in \Gamma
%% \and
%% \Gamma \vdash \overline{Var} : \overline{U}
%% %
%% }{\Gamma \vdash f(\overline{Var}) : \textbf{T}}
%% \end{mathpar}
\begin{mathpar}
\inferrule*[left=FunCall(\textup{f})]{
%
\Gamma \vdash f: (U_1, \ldots, U_n) \rightarrow \textbf{T}
\and
\Gamma \vdash x_1 : U_1
\and
\ldots
\and
\Gamma \vdash x_n : U_n
%
}{\Gamma \vdash f(x_1, \ldots, x_n) : \textbf{T}}
\end{mathpar}



% int BinOp
 \begin{mathpar}
\inferrule*[left={BinOp(\textup{e1,e2},Op=\{ADD,SUB,MUL,DIV,MOD,OR,AND,GT,LT,GE,LE\})}]{\Gamma \vdash e_1 : \textbf{int} \\ \vdash e_2 : \textbf{int}}{\Gamma \vdash e_1 Op\ e_2 : \textbf{int}}
\end{mathpar}

% other BinOp
 \begin{mathpar}
\inferrule*[left={BinOp(\textup{e1,e2},Op=\{NE,EQ\})}]{\Gamma \vdash e_1 : \textbf{T} \notin \{\textbf{StructType, ArrayType, void}\} \\ \vdash e_2 : \textbf{T}}{\Gamma \vdash e_1\ Op\ e_2 : \textbf{int}}
\end{mathpar}


% ArrayAccessExpr
\begin{mathpar}
\inferrule*[left={ArrayAccessExpr(\textup{e1,e2})}]{\Gamma \vdash e_1 : \textbf{T} \in \{\textbf{ArrrayType}_{\textbf{elemType}},\textbf{PointerType}_{\textbf{elemType}}\} \\ \vdash e_2 : \textbf{int}}{\Gamma \vdash e_1[e_2] : \textbf{elemType} }
\end{mathpar}

% \begin{mathpar}
% \inferrule*[left={ArrayAccessExpr}]{\Gamma \vdash Expr_1 : \textbf{AT} \in \{ArrrayType_elemType,PointerType_elemType} \\ \vdash Expr_2 : \textbf{int}}{\Gamma \vdash Expr_1[Expr_2] : \textbf{elemType} }
% \end{mathpar}

% FieldAccessExpr
\begin{mathpar}
\color{red}
\inferrule*[left={FieldAccessExpr(\textup{e,fieldName})}]{\Gamma \vdash e : \textbf{ClassType, StructType}_{fieldName : \textbf{T}}}{\Gamma \vdash e.fieldName : \textbf{T} }
\end{mathpar}


% ClassFunctionCallExpr
\begin{mathpar}
\color{red}
\inferrule*[left={ClassFunCallExpr(\textup{e,f})}]{
\Gamma \vdash e : \textbf{ClassType}
\and
\Gamma \vdash f: (U_1, \ldots, U_n) \rightarrow \textbf{T}
\and
\Gamma \vdash x_1 : U_1
\and
\ldots
\and
\Gamma \vdash x_n : U_n
%
}{\Gamma \vdash e.f(x_1, \ldots, x_n) : \textbf{T}}
\end{mathpar}

% ValueAtExpr
\begin{mathpar}
\inferrule*[left={ValueAtExpr(\textup{e})}]{\Gamma \vdash e : \textbf{PointerType}_\textbf{T}}{\Gamma \vdash *e : \textbf{T} }
\end{mathpar}

% AddressOfExpr
\begin{mathpar}
\inferrule*[left={AddressOfExpr(\textup{e})}]{\Gamma \vdash e : \textbf{T}}{\Gamma \vdash \&e : \textbf{PointerType}_\textbf{T} }
\end{mathpar}


% SizeOfExpr
\begin{mathpar}
\inferrule*[left={SizeOf(\textup{t})}]{\ }{\Gamma \vdash \ sizeof(t) : \textbf{int} }
\end{mathpar}

% TypeCastExpr
\begin{mathpar}
\inferrule*[left={TypeCastExpr(char to int)}]{\Gamma \vdash e : \textbf{char}}{\Gamma \vdash \ (\textbf{int})e : \textbf{int} }
\end{mathpar}


% TypeCastExpr for class
\begin{mathpar}
\color{red}
\inferrule*[left={TypeCastExpr(subclass to ancestor class)}]{\Gamma \vdash e : \textbf{T}_\textbf{1} \\ \textbf{T}_\textbf{1} <: \textbf{T}_\textbf{2}}{\Gamma \vdash \ (\textbf{T}_\textbf{2})e : \textbf{T}_\textbf{2} }
\end{mathpar}


\begin{mathpar}
\inferrule*[left={TypeCastExpr(array to pointer)}]{\Gamma \vdash e : \textbf{ArrayType}_\textbf{elemType}}{\Gamma \vdash \ (\textbf{*elemType})e : \textbf{PointerType}_\textbf{elemType} }
\end{mathpar}

\begin{mathpar}
\inferrule*[left={TypeCastExpr(pointer to pointer)}]{\Gamma \vdash e : \textbf{PointerType}_\textbf{elemType1}}{\Gamma \vdash \ (\textbf{*elemType2})e : \textbf{PointerType}_\textbf{elemType2} }
\end{mathpar}


\section{Statements}


\begin{mathpar}
\inferrule*[left={While}]{\Gamma \vdash e : \textbf{int}}{\Gamma \vdash while(e)\ s}
\end{mathpar}

\begin{mathpar}
\inferrule*[left={If(no else)}]{\Gamma \vdash e : \textbf{int}}{\Gamma \vdash if(e)\ s}
\and
\inferrule*[left={If(with else)}]{\Gamma \vdash e : \textbf{int}}{\Gamma \vdash if(e)\ s_1\ else\ s_2}
\end{mathpar}


\begin{mathpar}
\inferrule*[left=Assign]{\Gamma \vdash e_1 : \textbf{T} \notin \{\textbf{void, ArrayType}\} \\ \Gamma \vdash e_2 : \textbf{T}}{\Gamma \vdash e_1 = e_2}
\end{mathpar}

\begin{mathpar}
\inferrule*[left=Return(from \textup{f})]{\Gamma \vdash f: (U_1, \ldots, U_n) \rightarrow \textbf{T} \\ \Gamma \vdash e : \textbf{T}}{\Gamma \vdash return\ e}
\end{mathpar}

\begin{mathpar}
\inferrule*[left=Return(nothing from \textup{f})]{\Gamma \vdash f: (U_1, \ldots, U_n) \rightarrow \textbf{void}}{\Gamma \vdash return\ \varnothing}
\end{mathpar}
 
\end{document}
